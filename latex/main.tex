\documentclass[]{article}
\usepackage{amssymb,amsmath}

\makeatother
\IfFileExists{xurl.sty}{\usepackage{xurl}}{} % add URL line breaks if available
\IfFileExists{bookmark.sty}{\usepackage{bookmark}}{\usepackage{hyperref}}
\hypersetup{
  pdftitle={FM-project-1},
  pdfauthor={Kexing Ying},
  colorlinks=true,
  linkcolor=Maroon,
  filecolor=Maroon,
  citecolor=Blue,
  urlcolor=red,
  pdfcreator={LaTeX via pandoc}}
\urlstyle{same} % disable monospaced font for URLs

\usepackage{fontspec}
  \setmainfont{Bitstream Charter}
  \setmonofont{Fira Code}
\usepackage[utf8x]{inputenc}
\usepackage{amssymb}

\usepackage[margin = 1.5in]{geometry}
\usepackage{graphicx}
\usepackage{amsthm}
\usepackage{mathtools}
\usepackage[ruled,vlined]{algorithm2e}

\usepackage{minted}
\usepackage{newunicodechar}
\newunicodechar{Ñ}{\ensuremath{\mathcal{N}}}
\newunicodechar{ᵐ}{\ensuremath{^m}}
\newunicodechar{ᶜ}{\ensuremath{^c}}

\theoremstyle{definition}
\newtheorem{theorem}{Theorem}
\newtheorem{definition}{Definition}[section]
\newtheorem{lemma}{Lemma}[section]
\newtheorem{proposition}{Proposition}[section]

\newcommand\eqae{\mathrel{\overset{\makebox[0pt]{\mbox{\normalfont\tiny\sffamily a.e.}}}{=}}}

\setlength\parindent{0pt}

\title{Formalising Mathematics\\
  \large Project 1}
\author{Kexing Ying}

\begin{document}
\maketitle

\section*{Introduction}

For the first project in \textit{Formalising Mathematics}, I decided to formalise a result from 
measure theory known as Egorov's theorem. Egorov's theorem is a useful result establishing a 
relation between convergence almost everywhere and uniform convergence. In particular, 
Egorov's theorem states that a sequence of almost everywhere convergent functions 
converge uniformly everywhere except on an arbitrarily small set.

Egorov's theorem is used to prove the Vitali convergence theorem 
(a generalisation of the monotone convergence theorem for uniformly integrable functions)
and is also useful to simplify more elementary results such as convergence almost 
everywhere implies convergence in measure. 

\section*{Formalisation of Egorov's Theorem}

We will in this section outline the proof of Egorov's theorem and comment on the formalisation effort.
For the remainder of this document, let us assume \(\alpha\) is a measure space with measure \(\mu\) 
and \(\beta\) is a second-countable metric space.

\begin{theorem}[Egorov's Theorem]
  If \((f_n : \alpha \to \beta)_{n = 0}^\infty\) is a sequence of measurable functions which 
  converge almost everywhere on a set \(s \subseteq \alpha\) of finite measure to 
  \(g : \alpha \to \beta\), then, for any \(\epsilon > 0\), there exists some \(t \subseteq s\) 
  with measure \(\mu(t) \le \epsilon\) and \(f_n\) converges uniformly to \(g\) on 
  \(s \setminus t\). 
\end{theorem}
\begin{proof}
  Let \(\epsilon > 0\) and for \(i, j \ge 1\), define 
  \[C_{ij} := \bigcup_{k = j}^\infty \{x \in s \mid |f_k(x) - g(x)| > i^{-1}\}.\]
  Since \(C_{ij}\) is measurable with finite measure and \(C_{i, j + 1} \subseteq C_{i, j}\), by the 
  continuity of measures from above, we have 
  \[\lim_{j \to \infty} \mu(C_{ij}) = 
    \mu\left(\bigcap_{j = 1}^\infty C_{ij}\right) = 0.\]
  Now, since \(f_k\) converges to \(g\) almost everywhere, by the definition of 
  limits, there exists a subsequence \((C_{i, J(i)})\) such that \(\mu(C_{i, J(i)}) < 
  \epsilon 2^{-i}\). Then, by defining 
  \[t := \bigcup_{i = 1}^\infty C_{i, J(i)},\]
  we have 
  \[\mu(s \setminus t) = \mu\left(\bigcup_{i = 1}^\infty C_{i, J(i)}\right) \le \sum_{i = 1}^\infty \mu(C_{i, J(i)})
    < \sum_{i = 1}^\infty \epsilon 2^{-i} = \epsilon.\]
  Now, by construction, the elements of \(t\) are namely the elements \(x\) 
  which \(|f_k(x) - g(x)|\) converges slower than \(i^{-1}\), the elements in 
  \(s \setminus t\) converges at a speed uniform bounded above by \(i^{-1}\), and 
  hence, \(f\) converges to \(g\) uniformly on \(s \setminus t\) as required.
\end{proof}

In Lean, this statement is represented by 
\begin{minted}[mathescape, numbersep=5pt, frame=lines, framesep=2mm]{Lean}
lemma tendsto_uniformly_on_of_ae_tendsto' [is_finite_measure μ]
  (hf : ∀ n, measurable (f n)) (hg : measurable g)
  (hfg : ∀ᵐ x ∂μ, tendsto (λ n, f n x) at_top (Ñ (g x)))
  {ε : ℝ} (hε : 0 < ε) :
∃ t, measurable_set t ∧ μ t ≤ ennreal.of_real ε ∧
  tendsto_uniformly_on f g at_top tᶜ
\end{minted}




\end{document}