\documentclass[]{article}
\usepackage{amssymb,amsmath}

\makeatother
\IfFileExists{xurl.sty}{\usepackage{xurl}}{} % add URL line breaks if available
\IfFileExists{bookmark.sty}{\usepackage{bookmark}}{\usepackage{hyperref}}
\hypersetup{
  pdftitle={FM-project-1},
  pdfauthor={Kexing Ying},
  colorlinks=true,
  linkcolor=Maroon,
  filecolor=Maroon,
  citecolor=Blue,
  urlcolor=red,
  pdfcreator={LaTeX via pandoc}}
\urlstyle{same} % disable monospaced font for URLs

\usepackage{fontspec}
  \setmainfont{Bitstream Charter}
  \setmonofont{Fira Code}
\usepackage[utf8x]{inputenc}
\usepackage{amssymb}

\usepackage[margin = 1.5in]{geometry}
\usepackage{graphicx}
\usepackage{amsthm}
\usepackage{mathtools}
\usepackage[ruled,vlined]{algorithm2e}
\usepackage{parskip}

\usepackage{minted}
\usepackage{newunicodechar}
\newunicodechar{Ñ}{\ensuremath{\mathcal{N}}}
\newunicodechar{ᵐ}{\ensuremath{^m}}
\newunicodechar{ᶜ}{\ensuremath{^c}}

\theoremstyle{definition}
\newtheorem{theorem}{Theorem}
\newtheorem{definition}{Definition}[section]
\newtheorem{lemma}{Lemma}[section]
\newtheorem{proposition}{Proposition}[section]

\newcommand\eqae{\mathrel{\overset{\makebox[0pt]{\mbox{\normalfont\tiny\sffamily a.e.}}}{=}}}

\title{Formalising Mathematics\\
  \large Project 1}
\author{Kexing Ying}

\begin{document}
\maketitle

\section*{Introduction}

For the first project in \textit{Formalising Mathematics}, I decided to formalise a result from 
measure theory known as Egorov's theorem. Egorov's theorem is a useful result establishing a 
relation between convergence almost everywhere and uniform convergence. In particular, 
Egorov's theorem states that a sequence of almost everywhere convergent functions 
converge uniformly everywhere except on an arbitrarily small set.

Egorov's theorem is used to prove the Vitali convergence theorem 
(a generalisation of the monotone convergence theorem for uniformly integrable functions)
and is also useful to simplify more elementary results such as convergence almost 
everywhere implies convergence in measure. 

The content of this project has now been published to Lean's maths library - 
\texttt{mathlib} and can be found (at the time of writing) 
\href{https://github.com/leanprover-community/mathlib/blob/master/src/measure_theory/function/uniform_integrable.lean}{here}.
As the name of the \texttt{mathlib} file suggest, this theorem is intended for future 
work and is a part of the overall 
\href{https://github.com/leanprover-community/mathlib/projects/15}{project} 
for formalising martingales in Lean others and I are working towards. 


\section*{Almost Everywhere Filter}

In order to formalise Egorov's theorem, we will need to use the almost everywhere 
filter from \texttt{mathlib} in order to state the statement of convergence everywhere. 
We will in this section quickly introduce this definition. 

\texttt{mathlib} defines the almost everywhere filter as the set of all 
co-null sets (set which complement has measure zero), namely it is defined as
(omitting the proofs)
\begin{minted}[mathescape, numbersep=5pt, frame=lines, framesep=2mm, fontsize=\small]{Lean}
def measure.ae {α} [measurable_space α] (μ : measure α) : filter α :=
{ sets := {s | μ sᶜ = 0}, ... }
\end{minted}
With the above filter defined, one may introduce all the standard definitions involving 
almost everywhere with this filter. In our use case, we would like to state the 
condition of convergence almost everywhere. To achieve this, we use the definition
\begin{minted}[mathescape, numbersep=5pt, frame=lines, framesep=2mm, fontsize=\small]{Lean}
def eventually (p : α → Prop) (f : filter α) : Prop := {x | p x} ∈ f
\end{minted}
Unfolding the definition, we observe that \mintinline{Lean}{eventually p μ.ae} is 
the proposition that \(\mu(\{x \mid \mathtt{p}(x)\})^c = 0\), namely, 
the set of elements not satisfying \texttt{p} has measure 0. For ease of use, 
we denote the statement \mintinline{Lean}{eventually p μ.ae} in 
as \mintinline{Lean}{∀ᵐ x ∂μ, p x}. Thus, with this in mind, we may formulate 
convergence almost everywhere by taking \texttt{p} the proposition that the 
sequence of function converges at the point \(x\).

In particular, the proposition that the sequence of functions \(f_n\) converges 
\(g\) almost everywhere is represented by the statement 
\begin{minted}[mathescape, numbersep=5pt, frame=lines, framesep=2mm, fontsize=\small]{Lean}
  ∀ᵐ x ∂μ, x ∈ s → tendsto (λ n, f n x) at_top (Ñ (g x))
\end{minted}

\section*{Formalisation of Egorov's Theorem}

We will in this section outline the proof of Egorov's theorem and comment on the formalisation effort.
For the remainder of this document, let us assume \(\alpha\) is a measure space with measure \(\mu\) 
and \(\beta\) is a second-countable metric space.

\begin{theorem}[Egorov's Theorem]
  If \((f_n : \alpha \to \beta)_{n = 0}^\infty\) is a sequence of measurable functions which 
  converge almost everywhere on a set \(s \subseteq \alpha\) of finite measure to 
  \(g : \alpha \to \beta\), then, for any \(\epsilon > 0\), there exists some \(t \subseteq s\) 
  with measure \(\mu(t) \le \epsilon\) and \(f_n\) converges uniformly to \(g\) on 
  \(s \setminus t\). 
\end{theorem}
\begin{proof}
  Let \(\epsilon > 0\) and for \(i, j \ge 1\), define 
  \[C_{ij} := \bigcup_{k = j}^\infty \{x \in s \mid |f_k(x) - g(x)| > i^{-1}\}.\]
  Since \(C_{ij}\) is measurable with finite measure and \(C_{i, j + 1} \subseteq C_{i, j}\), by the 
  continuity of measures from above, we have 
  \[\lim_{j \to \infty} \mu(C_{ij}) = 
    \mu\left(\bigcap_{j = 1}^\infty C_{ij}\right) = 0.\]
  Now, since \(f_k\) converges to \(g\) almost everywhere, by the definition of 
  limits, there exists a subsequence \((C_{i, J(i)})\) such that \(\mu(C_{i, J(i)}) < 
  \epsilon 2^{-i}\). Then, by defining 
  \[t := \bigcup_{i = 1}^\infty C_{i, J(i)},\]
  we have 
  \[\mu(t) = \mu\left(\bigcup_{i = 1}^\infty C_{i, J(i)}\right) \le 
    \sum_{i = 1}^\infty \mu(C_{i, J(i)})
    < \sum_{i = 1}^\infty \epsilon 2^{-i} = \epsilon.\]
  Now, by construction, the elements of \(t\) are namely the elements \(x\) 
  which \(|f_k(x) - g(x)|\) converges slower than \(i^{-1}\), the elements in 
  \(s \setminus t\) converges at a speed uniform bounded above by \(i^{-1}\), and 
  hence, \(f\) converges to \(g\) uniformly on \(s \setminus t\) as required.
\end{proof}

In Lean, Egorov's theorem is represented by 
\begin{minted}[mathescape, numbersep=5pt, frame=lines, framesep=2mm, fontsize=\small]{Lean}
theorem tendsto_uniformly_on_of_ae_tendsto
  (hf : ∀ n, measurable (f n)) (hg : measurable g) 
  (hsm : measurable_set s) (hs : μ s ≠ ∞)
  (hfg : ∀ᵐ x ∂μ, x ∈ s → tendsto (λ n, f n x) at_top (Ñ (g x))) 
  {ε : ℝ} (hε : 0 < ε) :
∃ t ⊆ s, measurable_set t ∧ μ t ≤ ennreal.of_real ε ∧ 
  tendsto_uniformly_on f g at_top (s \ t)
\end{minted}
which immediately, by choosing \(s = \mathtt{set.univ}\), we obtain a version of Egorov's 
theorem for finite measures (which is arguably more useful especially in probability theory)
\begin{minted}[mathescape, numbersep=5pt, frame=lines, framesep=2mm, fontsize=\small]{Lean}
lemma tendsto_uniformly_on_of_ae_tendsto' [is_finite_measure μ]
  (hf : ∀ n, measurable (f n)) (hg : measurable g)
  (hfg : ∀ᵐ x ∂μ, tendsto (λ n, f n x) at_top (Ñ (g x)))
  {ε : ℝ} (hε : 0 < ε) :
∃ t, measurable_set t ∧ μ t ≤ ennreal.of_real ε ∧
  tendsto_uniformly_on f g at_top tᶜ
\end{minted}

To prove this statement several auxiliary definitions were introduced. These are 
namely,
\begin{itemize}
  \item \mintinline{Lean}{measure_theory.egorov.not_convergent_seq};
  \item \mintinline{Lean}{measure_theory.egorov.not_convergent_seq_lt_index};
  \item \mintinline{Lean}{measure_theory.egorov.Union_not_convergent_seq},
\end{itemize}
which corresponds to the declarations \(C_{ij}, J(i)\) and \(\bigcup_{i = 1}^\infty C_{i, J(i)}\) 
in the proof above respectively. The implementation of these definitions are mostly
identical to their mathematical definitions except perhaps the definition of 
\mintinline{Lean}{not_convergent_seq_lt_index} where we explicitly invoke 
the axiom of choice. 

Recalling that \mintinline{Lean}{not_convergent_seq_lt_index} provides the index 
of the subsequence \(J(i)\) of \(C_{ij}\) such that \(\mu(C_{i,J(i)}) < \epsilon 2^{-i}\)
which existence is provided by \(\lim_{j \to \infty}(\mu(C_{ij})) = 0\), to construct 
such a function, one needs to use the axiom of choice. More specifically, 
we will use \mintinline{Lean}{classical.some} which depends on the axiom 
\mintinline{Lean}{classical.choice}.
\begin{minted}[mathescape, numbersep=5pt, frame=lines, framesep=2mm, fontsize=\small]{Lean}
  def some {a : Sort*} {p : a → Prop} (h : ∃ x, p x) : a := ...
\end{minted}
In words, given some predicate on some type and the proof that there exists a term 
of that type satisfying the predicate, \mintinline{Lean}{classical.some} chooses an 
term of that type satisfying the predicate. Hence, armed with \mintinline{Lean}{classical.some},
we can construct the required index function by providing a proof 
that there exists some \(J_i\) such that \(\mu(C_{i, J_i}) < \epsilon 2^{-i}\).
This proof is known as \mintinline{Lean}{measure_theory.egorov.exists_not_convergent_seq_lt}.

With these definitions in mind, the proof of the overall statement in Lean is 
extremely similar to the proof on paper except perhaps for the fact that \(f\)
converges uniformly on \(s \setminus t\). While on paper, we compared the rate 
of convergence, we will need to fill in the details in Lean. Nonetheless, by 
unfolding the definitions, this remains to be a routine \(\epsilon-\delta\) proof.

\section*{Implementation Decisions}

There are in general two ways of constructing proofs in analysis such as this one. 
One may, as this project has demonstrated, define auxiliary definitions and lemmas 
for the main theorem and prove it modularly. This approach is more readable and 
is in general easier to work with although it does generate many unuseful 
declarations. One may make these declarations private although this is normally 
discouraged since it is possible one might want to reuse them in other contexts, 
instead, one normally put these declarations in a special namespace to avoid 
name clashes in the future.

This approach is however not always possible especially if the proof involves 
inter-winding definitions which carries along a lot of data and proofs. This can 
be seen from the fact that all lemmas and definitions in this project require 
a lot of arguments which results in clutter. To avoid this, one may simply 
define local declarations within the proof itself as all required arguments are 
then within the local context. 

An example of the second approach can be found in \texttt{mathlib} in the formalisation 
of the Lebesgue decomposition theorem 
\href{https://github.com/leanprover-community/mathlib/blob/master/src/measure_theory/decomposition/lebesgue.lean#L604}{here}
where we see that \(\xi\) is defined locally with local lemmas. The downside to this 
approach is that the proofs becomes excessively long, and normally becomes hard 
to maintain as the proof will take much longer to compile after each tactic.  

\end{document}